A {\bf linear list} is a sequence of $n\geq 0$ nodes {\tt X[1],X[2],$\ldots$,X[n]} whose essential structural properties involve only the relative positions between  items as they appear in a line. The only thing we care about in such structures are the facts that, if $n>0$, {\tt X[1]} is the first node and {\tt X[n]} is the last node; and if $1<k<n$, the $k$th node {\tt X[k]} is preceded by {\tt X[k-1]} and followed by {\tt X[k+1]}.

\vskip 1mm
The operations we might want to perform on linear lists include:

\vskip 2mm
{\rm i)}\qquad Gain access to the $k$th node of the list to examine and/or to change the contents of its fields.
\vskip 2mm
{\rm ii)}\qquad Delete the $k$th node.

\vskip 2mm
{\rm iii)}\qquad Delete the $k$th node.

\vskip 2mm
{\rm iv)}\qquad Combine two or more linear lists into a single list.

\vskip 2mm
{\rm v)}\qquad Split a linear list into two or more lists.

\vskip 2mm
{\rm vi}\qquad Make a copy of a linear list.

\vskip 2mm
{\rm vii}\qquad Determine the number of nodes in a list.

\vskip 2mm
{\rm viii}\qquad Sort the nodes of the list into ascending order based on certain fields of the nodes.

\vskip 2mm
{\rm ix}\qquad Search the list for the occurrence of a node with a particular value in some field.

\vskip 3mm
In operations {\rm (i), (ii), (iii)} the special cases $k=1$ and $k=n$ are of principal importance, since the first and last iteams of a linear list may be easier to get at than a general element is.

\vskip 3mm
Linear list in which insertions, deletions, and accesses to values occur almost always at the first or last node are frequently encountered, and we give them special names:

\vskip 2mm
A {\bf stack} is a linear list for which all insertions and deletions (and usually all accesses) are made at one end of the list.

\vskip 2mm
A {\bf queue} is a linear list for which all insertions are made at one end of the list; all deletions (and usually all accesses) are made at the other end.

\vskip 2mm
A {\bf deque} ("double-ended queue") is a linear list for which all insertions and deletions (and usually all accesses) are made at the ends of the list.


\vfill\eject
\bye
